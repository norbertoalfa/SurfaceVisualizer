% !TeX root = ../libro.tex
% !TeX encoding = utf8

\chapter{Guía de uso del programa}\label{ap:apendice2}

En este capítulo se explicará brevemente cómo utilizar el programa. Aun así, la interfaz del programa se ha intentado diseñar lo más sencilla y clara posible, mostrando adicionalmente cuadros de texto si se mantiene el ratón sobre ciertos elementos.\\
\\Si sólo quiere hacer uso del programa ``procesador'', ejecute el comando:
\begin{center}
 make read FILE=<archivo>
\end{center}
pero desde el directorio ``procesador''. Esta acción devolverá la traducción de ``<archivo>'' al fichero ``../shaders/functions.s'' y la salida de error en ``../error.log''.\\
\\Si queremos utilizar el programa completo, primero instalaremos el programa tal y como se indica en el apéndice de \textbf{Instalación del software}. Una vez instalado, se abrirá siempre con el comando ``make'', ``make execute'' o equivalentemente ``./bin/program''. Teniendo abierto el programa se mostrará siempre la última parametrización compilada con éxito. En el lateral izquierdo aparecerá un elemento de la interfaz, el menú, donde se podrá:
\begin{itemize}
	\item Parametrización:
	\begin{itemize}
		\item Seleccionar una parametrización ya existente, crearla o compilarla. También se podrá editar la actual, en cuyo caso aparecerá una ventana de edición de la propia interfaz. Dicha ventana también aparecerá en caso de que hayan errores léxicos, sintácticos o semánticos en la parametrización, con la salida del error desplegada. Como ejemplos se aportarán parametrizaciones de una gran variedad de superficies, ubicadas en el directorio ``variedades''.
		\item Cambiar el tamaño de la malla de partida, requeriendo su posterior actualización manual (botón contiguo).
		\item Visualizar la ventana con los parámetros temporales, con un tick que indica si está activa la ventana o no. Estará semi-visible si la superficie no tiene parámetros temporales. Dicha ventana mostrará los parámetros temporales en orden, permitiendo moverlos manualmente o generar una animación:
		\begin{itemize}
			\item Sin: movimiento sinusoidal entre el valor $0$ y $1$. Ideal para animaciones oscilantes.
			\item Lineal: movimiento lineal del valor $0$ al $1$, volviendo instantáneamente al $0$. Utilizado para movimientos lineales respecto al tiempo, que junto con el uso de funciones periódicas se puede generar la sensación de movimiento infinito (como los ejemplos wavesX.in).
		\end{itemize}
	\end{itemize}
	\item Visualización del objeto:
	\begin{itemize}
		\item Invertir normales (si no se quiere modificar la parametrización).
		\item Visualizar en modo malla (``Poligon mode'').
		\item Activar/desactivar la auto-rotación (rotación entorno al punto hacia el que mira la cámara).
		\item Ver los vectores tangentes, bitangentes y normales a los puntos (ya sean los de la malla inicial o de todos los generados).
		\item Cambiar modo del color de la superficie, ya sea el color base, la curvatura de Gauss, el área diferencial, la altura o los puntos críticos de ésta vista como función de Morse. Una vez seleccionado un modo, aparecerán coeficientes que permitirán ajustar correctamente la visualización a la superficie actual.
	\end{itemize}
	\item Opciones del teselado:
	\begin{itemize}
		\item Desactivar/activar el teselado.
		\item Indicar la precisión a la que se desea llegar con el teselado.
		\item Opciones avanzadas: modificar aquellos coeficientes específicos del teselado, como el tipo de mejora de rendimiento a usar (``improve'' normal o específica), la distancia de teselado, el umbral para detectar bordes y el exponente aplicado a la curvatura de Gauss. Todos ellos se inician con un valor por defecto.
	\end{itemize}
	\item Iluminación: es posible cambiar los coeficientes del modelo de iluminación ``Phong'' y ver el vector de dirección de la luz actualmente. La luz no es direccional, el vector indica la dirección de la luz con respecto al origen $(0,0)$.
	\item Estadísticas: muestra los fotogramas por segundo y la latencia medias, junto con el número de primitivas generadas tras la fase del geometry shader (después del tessellation shader). Permite además grabar los datos y los almacena de manera automática tras 22 segundos (antes si pulsamos ``Stop'' y seguidamente ``Save info'') en un fichero en el directorio raíz, con nombre dependiente de la parametrización y configuración actual.
\end{itemize}

\begin{figure}[h]
  	\centering
  	\includegraphics[width=0.9\textwidth]{interfaz1}
  	\caption{Ventanas iniciales.}
  	\label{fig:interfaz1}
\end{figure}
\newpage
\begin{figure}[h]
  	\centering
  	\includegraphics[width=0.9\textwidth]{interfaz2}
  	\caption{Ventana de edición de código.}
  	\includegraphics[width=0.9\textwidth]{interfaz3}
  	\caption{Ventanas de selección y creación de archivos.}
  	\label{fig:interfaz23}
\end{figure}
\newpage
Para mayor comodidad, se han incorporado atajos de teclado para ciertas acciones:
\begin{itemize}
	\item 'LCtrl' + 'R': activa/desactiva el modo de rotación automática.
	\item 'LCtrl' + 'P': cambia entre los modos de visualización malla y relleno.
	\item 'LCtrl' + 'N': activa/desactiva la visualización de normales.
	\item 'LCtrl' + 'L': ejecuta el ``procesador'' y compila los shaders.
\end{itemize}

Además, existen varios controles del ratón para manejar la cámara:
\begin{itemize}
	\item Botón izquierdo: si se mantiene pulsado y arrastramos, la cámara realizará una traslación en la dirección contraria.
	\item Botón derecho: si se mantiene pulsado y arrastramos, la cámara realizará una rotación (cámara orbital) en la dirección contraria.
	\item Rueda: controla el zoom.
	\item Botón de la rueda: reinicia la posición de la cámara.
\end{itemize}

\endinput
%------------------------------------------------------------------------------------
% FIN DEL APÉNDICE. 
%------------------------------------------------------------------------------------
