% !TeX root = ../libro.tex
% !TeX encoding = utf8
%
%*******************************************************
% Introducción
%*******************************************************

% \manualmark
% \markboth{\textsc{Introducción}}{\textsc{Introducción}} 

\chapter{Objetivos}

A continuación se mostrarán los objetivos inicialmente propuestos, los alcanzados y aquellos campos de estudio más utilizados.\\
\\Objetivos y métodos inicialmente propuestos:
\begin{itemize}
	\item Ámbito de las Matemáticas:
	\begin{itemize}
		\item Se tratarán algunos aspectos básicos de topología diferencial en dimensión baja.
		\item Se abordará una prueba más sencilla de un teorema clásico de Munkres sobre la unicidad de las estructuras diferenciables soportadas por una superficie topológica.
	\end{itemize}
	\item Ámbito de la Informática:
	\begin{itemize}
		\item Desarrollar software de visualización y animación $3$D de superficies, que permitirá ilustrar visualmente algunos conceptos matemáticos del ámbito de las superficies topológicas diferenciables.
		\item Se analizarán los algoritmos relacionados descritos en la literatura y se implementarán los más adecuados a los conceptos a ilustrar.
		\item El software debe ser eficiente, robusto y portable.
	\end{itemize}
\end{itemize}
Objetivos alcanzados y métodos usados:
\begin{itemize}
	\item Ámbito de las Matemáticas:
	\begin{itemize}
		\item Se han tratado algunos aspectos básicos de topología diferencial en dimensión baja. Para su correcto tratamiento han sido necesarias algunas herramientas provenientes de la teoría de Morse, que hemos utilizado sin incluir sus demostraciones. Han sido necesarios algunos conocimientos de análisis complejo (esfera de Riemann) y álgebra (para trabajar con los grupos fundamentales).
		\item Se ha abordado una prueba más sencilla de un teorema clásico de Munkres sobre la unicidad de las estructuras diferenciables soportadas por una superficie topológica. Sólo se ha alcanzado para el caso de superficies sin borde.
	\end{itemize}
	\item Ámbito de la Informática:
	\begin{itemize}
		\item Se ha desarrollado un software de visualización y animación $3$D de superficies definidas por cartas, que permitirá ilustrar visualmente conceptos matemáticos como el cálculo de normales, la curvatura de Gauss, funciones de Morse (sólo función altura) y sus puntos críticos.
		\item El usuario puede escribir cualquier función, que pueda definir con el lenguaje, que seguidamente será traducida a un lenguaje de programación (GLSL) el cual se compilará para ejecutarse en la GPU.
		\item Se han analizado los algoritmos de generación de árboles de expresión y árboles derivados, junto con algoritmos de teselado y procedimientos de muestreo. Se han implementado los más adecuados para la definición escogida de ``buena aproximación'' a una superficie, con el objetivo de mejorar la calidad visual con un buen rendimiento.
		\item El software realiza absolutamente todos los cálculos relativos a la teselación en la GPU, consiguiendo por tanto visualizar animaciones fluidas en tiempo real, con tiempos por cuadro del orden de milisegundos. Su portabilidad está sujeta a las librerías utilizadas, por lo que si es posible instalarlas con versiones iguales o superiores, será viable su instalación y ejecución.
	\end{itemize}
\end{itemize}
Campos de estudio más utilizados:
\begin{itemize}
	\item Ámbito de las Matemáticas:
	\begin{itemize}
		\item Topología diferencial en dimensión baja, análisis complejo y álgebra.
	\end{itemize}
	\item Ámbito de la Informática:
	\begin{itemize}
		\item Para el procesador: Procesadores de Lenguajes y Programación.
		\item Para el programa de visualización: Informática gráfica, Diseño Orientado a Objetos, Sistemas Operativos y algunos conceptos de Visión por Computador.
	\end{itemize}
\end{itemize}

\endinput
