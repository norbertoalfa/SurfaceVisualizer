\documentclass[11pt]{article}
\usepackage[utf8]{inputenc}
\usepackage{amsthm}
\usepackage{amssymb}
%Gummi|065|=)
\title{\textbf{Estructuras diferenciables sobre una superfície topológica}}
\author{Norberto Fernández de la Higuera}
\date{10/10/2020}

\newtheorem{lema}{Lema}
\newtheorem{teor}{Teorema}
\newtheorem{cor}{Corolario}
\newtheorem{prop}{Proposición}
\newtheorem*{teora}{Teorema A}
\newtheorem*{teorb}{Teorema B}

\begin{document}

\maketitle

\section{Introducción}

	\begin{teora}
		Toda variedad topológica tiene una estructura diferenciable.
	\end{teora}

	\begin{teorb}
		Todo homeomorfismo entre variedades diferenciables es isotópico a un difeomorfismo.
	\end{teorb}

	\begin{cor} (Teorema clásico de Munkres)
		Toda variedad topológica tiene una única estructura diferenciable salvo difeomorfismos.
	\end{cor}

\section{Resultados previos}

	\begin{teor} (de "alisamiento de asas")
		Sea S una variedad diferenciable, entonces:
		\begin{enumerate}
			\item Un embebimiento $\mathbb{R}^2 \rightarrow S$ puede isotoparse a un embebimiento diferenciable entorno al origen, quedando fijo fuera de un entorno mayor al anterior.
			\item Un embebimiento $D^1\times\mathbb{R} \rightarrow S$ que es diferenciable entorno a $\partial D^1\times\mathbb{R}$ puede isotoparse a un embebimiento diferenciable entorno a $D^1\times 0$, quedando fijo fuera de un entorno mayor al anterior y cercano a $\partial D^1\times\mathbb{R}$.
			\item Un embebimiento $D^2 \rightarrow S$ que es diferenciable entorno a $\partial D^2$ puede isotoparse a un embebimiento diferenciable en todo $D^2$, quedando fijo en un entorno pequeño de $\partial D^2$.
		\end{enumerate}
	\end{teor}
	
	\begin{cor}
		El teorema anterior sigue siendo cierto para un abierto de $\mathbb{R}^2$ en vez de para todo $\mathbb{R}^2$.
	\end{cor}

\section{Demostración del Teorema A}

	\begin{teora}
		Toda variedad topológica tiene una estructura diferenciable.
	\end{teora}
	\begin{proof}[Demostración]
		Sea S una variedad topológica sin borde, podemos coger un sistema coordenado de cartas finito. Vamos a construir por inducción una estuctura diferenciable $U_n = \cup_{i\leq n}h_i(\mathbb{R}^2)$, que por ser un sistema coordenado su límite debe de ser S, probando así el resultado. Cabe destacar que cada $U_i$ contiene a todos los anteriores. \\
		\\ La inducción empieza tomando una carta cualquiera del sistema. Si se considera la variedad $U_1$ con el atlas $\{(h_1,U_1)\}$ (un subconjunto abierto de una variedad es una variedad), $h_1$ es diferenciable para ésta de forma trivial (se compone con la inversa y queda la identidad en $\mathbb{R}^2$).\\
		\\ Una vez arrancada la inducción, suponiendo cierto para el paso $n-1$ vamos a extender la diferenciabilidad de $U_{n-1}$ a $U_n$. Sea $W=h_n^{-1}(U_{n-1})=h_n^{-1}(U_{n-1}\cap h_n(V_n))$, que es un abierto de $\mathbb{R}^2$ por ser $h_n$ un homeomorfismo entre $V_n$ y su imagen ($V_n$ es el abierto donde se define $h_n$ como carta para la variedad $S$). \\
		\\ Tenemos $W\subset V_n$ abierto en $\mathbb{R}^2$, por el \textbf{Hecho 1} sabemos que existe una triangulación geométrica suya y que al ser abierto (no tiene borde) al ir acercándose al borde los triángulos tienden a ser puntos. Queremos aplicar el "Teorema de suavizado de asas" en los vértices de los triángulos, seguidamente en los lados y finalmente en el interior de cada uno (aplicar los 3 apartados del teorema de forma consecutiva), pero para ello es necesario partir de un embebimiento de $\mathbb{R}^2$:
		\begin{enumerate}
			\item Para cada vértice $p$, elegimos una bola abierta lo suficientemente pequeña de forma que sus cierres no se corten (lo hacemos para todos los vértices de 1 vez), $B(p,\varepsilon _p)$ que es difeomorfo a $\mathbb{R}^2$ ($f_p:\mathbb{R}^2\rightarrow B$ difeomorfismo, con $f_p(0)=p$). Tomamos $g=h_n\circ f_p:\mathbb{R}^2\rightarrow h_n(W)$, que es un embebimiento por serlo $h_n:B\rightarrow h_n(W)$ ($h_n$ restringida a $B$ en vez de $W$) y en particular, $f_p$. \\
			\\ Aplicamos el apartado 1 del Teorema y obtenemos una $\widehat{g}$ isotópica a la primera, que es diferenciable en $O$ (entorno abierto del origen, con $0=f_p^{-1}(p)$) y además queda fija fuera de otro entorno un poco mayor $O'\supset O$, con $\overline{f_p(O_p)}\subset B $. Si tomamos $h'_n|_{B}=\widehat{g}\circ f_p^{-1}$ y $h'_n|_{W-B}=h_n$, está bien definida porque en $B-f_p(O_p)$ al aplicar $f_p^{-1}$ nos lleva a $\mathbb{R}^2-O_p$, que es donde $\widehat{g}=g=h_n|_B\circ f_p$, es decir: \\
			$h'_n|_{\partial B}=\widehat{g}\circ(f_p^{-1}|_{\partial B})=g\circ(f_p^{-1}|_{\partial B})=h_n$, por lo que la función a trozos está bien definida, es diferenciable entorno a $p$ y no se altera fuera de $B$. \\
			\\ Éste paso se puede realizar de forma simultánea para todos los vértices, obteniendo así una $h'_n$ que es diferenciable entorno a todos los vértices y se mantiene $h_n$ fuera de un entorno de cada vértice, algo mayor que el anterior (entornos con cierres disjuntos). Es por ello que para no cargar demasiado la notación se llamará a esa nueva función $h_n$.
			\item Se busca de igual modo la forma de llevar un entorno de cada segmento a $\mathbb{R}^2$ con un difeomorfismo, aplicar el 2º apartado del teorema y de forma similar deshacer el embebimiento para obtener la $h_n$ (manteniendo la notación).
			\item Buscamos finalmente una curva dentro del "esqueleto" de entornos de los lados de los triángulos que junto con su componente interior sea difeomorfa a la bola cerrada unidad, aplicar el 3$^{er}$ apartado del teorema y así obtener una $h_n$ diferenciable en todo $W$.
		\end{enumerate}
		
		Cabe destacar que en el borde de $W$ la aplicación se queda intacta, ya que todo el rato se está trabajando en el interior de $W$ (es abierto) y en todo paso de la "suavización" de $h_n$ siempre se deja inalterado el espacio complemento de un entorno mayor al aquel donde se obtiene la diferenciabilidad. Es por ello que se puede extender el $h_n$ obtenido a todo $\mathbb{R}^2$, porque está bien definido. 
	\end{proof}

\section{Demostración del Teorema B}



\section{Hechos}



\end{document}
