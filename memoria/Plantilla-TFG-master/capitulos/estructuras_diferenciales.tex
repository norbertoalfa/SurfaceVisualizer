% !TeX root = ../libro.tex
% !TeX encoding = utf8

\chapter{Estructuras diferenciales}

\section{Resultados previos}

	\begin{teorema} (de ``alisamiento de asas'')
		Sea S una variedad diferenciable, entonces:
		\begin{enumerate}
			\item Un embebimiento $\mathbb{R}^2 \rightarrow S$ puede isotoparse a un embebimiento diferenciable entorno al origen, quedando fijo fuera de un entorno mayor al anterior.
			\item Un embebimiento $D^1\times\mathbb{R} \rightarrow S$ que es diferenciable entorno a $\partial D^1\times\mathbb{R}$ puede isotoparse a un embebimiento diferenciable entorno a $D^1\times 0$, quedando fijo fuera de un entorno mayor al anterior y cercano a $\partial D^1\times\mathbb{R}$.
			\item Un embebimiento $D^2 \rightarrow S$ que es diferenciable entorno a $\partial D^2$ puede isotoparse a un embebimiento diferenciable en todo $D^2$, quedando fijo en un entorno pequeño de $\partial D^2$.
		\end{enumerate}
	\end{teorema}
	
	\begin{corolario}
		El teorema anterior sigue siendo cierto para un abierto de $\mathbb{R}^2$ en vez de para todo $\mathbb{R}^2$.
	\end{corolario}


\section{Hechos utilizados para los teoremas}

\endinput
%------------------------------------------------------------------------------------
% FIN DEL CAPÍTULO. 
%------------------------------------------------------------------------------------
