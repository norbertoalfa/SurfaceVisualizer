% !TeX root = ../libro.tex
% !TeX encoding = utf8

\chapter{Instalación del software}\label{ap:apendice1}

\section{Requisitos previos}

Los requisitos mínimos para poder compilar el proyecto son los siguientes:
\begin{itemize}
	\item SO: Ubuntu $18.04$ LTS o superior (o distribuciones similares).
	\item GPU: compatible con versión de OpenGL $4.4$ o superior (para poder usar el tessellation shader, entre otros).
	\item Dependencias: ``make'', para poder usar el makefile, y ``apt'', para gestionar paquetes.
\end{itemize}

A continuación se muestran las dependencias del proyecto, aunque el propio makefile las comprobará y actualizará:
\begin{itemize}
	\item Procesador:
	\begin{itemize}
		\item gcc.
		\item flex.
		\item bison, versión $3.5$ o superior.
	\end{itemize}
	\item Visualizador de Superficies (además de las de Procesador):
	\begin{itemize}
		\item g++, versión $11$ o superior.
		\item libglapi-mesa, mesa-utils y mesa-common-dev, que es una implementación de código abierto de OpenGL.
		\item libglfw$3$ y libglfw$3$-dev, para que la aplicación pueda gestionar las ventanas del sistema.
	\end{itemize}
\end{itemize}

\section{Instalación}

Para instalar el programa basta con clonar el repositorio de GitHub y ejecutar la orden ``make install'' desde la terminal, en el directorio raíz del programa. Realizará la comprobación de las dependencias de forma automática y en caso de no estar instaladas solicitará la confirmación de su instalación (a la última versión). Después compilará automáticamente el programa.\\
\\ Internamente utiliza  ``apt-get upgrade'', excepto para Bison, que será instalado haciendo uso de recursos locales, con la versión exacta $3.5$.

\section{Errores de compilación}

En caso de que aparezcan errores al compilar, compruebe detenidamente que se cumplen los \textbf{Requisitos previos}. Es posible que algunas dependencias no estén actualizadas para el comando ``apt'' y sea necesario instalarlo manualmente desde sus respectivas páginas oficiales.

\endinput
%------------------------------------------------------------------------------------
% FIN DEL APÉNDICE. 
%------------------------------------------------------------------------------------
