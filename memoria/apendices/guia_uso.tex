% !TeX root = ../libro.tex
% !TeX encoding = utf8

\chapter{Guía de uso del programa}\label{ap:apendice2}

En este capítulo se explicará brevemente cómo utilizar el programa. Aun así, la interfaz del programa se ha intentado diseñar lo más sencilla y clara posible, mostrando adicionalmente cuadros de texto si se mantiene el ratón sobre ciertos elementos.\\
\\Primero iniciaremos el programa tal y como se indica en el apéndice de \textbf{Instalación del software}. Una vez abierto el programa se mostrará siempre la última parametrización compilada existosamente. En el lateral izquierdo aparecerá un elemento de la interfaz, el menú, donde se podrá:
\begin{itemize}
	\item Seleccionar la parametrización, ya sea una existente, crear una o compilar la actual.
	\item Cambiar el tamaño de la malla de partida, requeriendo su posterior actualización manual (botón contiguo).
	\item Visualizar la ventana con los parámetros temporales, con un tick que indica si está activa la ventana o no. Estará semi-visible si la superficie no tiene parámetros temporales. Dicha ventana mostrará los parámetros temporales en orden, permitiendo moverlos manualmente o generar una animación:
	\begin{itemize}
		\item Auto
	\end{itemize}
\end{itemize}

\endinput
%------------------------------------------------------------------------------------
% FIN DEL APÉNDICE. 
%------------------------------------------------------------------------------------
