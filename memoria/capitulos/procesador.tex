% !TeX root = ../libro.tex
% !TeX encoding = utf8

\chapter{Procesador}

En este capítulo nos centramos en el diseño de un lenguaje sencillo para representar las cartas que definen a la superficie, que es un punto clave de la aplicación.\\
\\El procesador del lenguje tendrá como salida lenguaje GLSL para compilarlo como un shader y así transformar la malla inicial (triangulación del conjunto $[0,1]\times[0,1] \subset \R^2$), dando forma a la superficie. Paralelamente construirá el árbol de expresión de cada parametrización, permitiendo el cálculo de las derivadas parciales, necesarias para la obtención de las normales y la curvatura de Gauss.\\
\\A continuación se muestra un código de ejemplo que ilustrará la forma de una parametrización utilizando el lenguaje deseado:

\begin{center}
\begin{tabular}{| p{14cm} |}
\hline
//-- Ejemplo para la parametrización de la esfera --//\\
// Definir las costantes (PI es una cte predefinida)\\
PI$2$ : real = $2$*PI;\\
\\
// Definir funciones adicionales (sin y cos están predefinidas)\\
compx(u, v : real) : real = cos(v*PI)*cos(u*PI$2$);\\
compy(u, v : real) : real = cos(v*PI)*sin(u*PI$2$);\\
compz(u, v : real) : real = sin(v*PI);\\
\\
// Definir la función principal\\
f(u, v : real) : vec$3$ = vec$3$(compx(u,v), compy(u,v), compz(u,v));\\
\\
// Usar otra función para redefinir el dominio, en vez del cuadrado $[0,1]\times[0,1]$\\
g(u, v : real, t$0$, t$1$ : real) : vec$3$ = f(t$0$ * (u-$0.5$), t$1$ * (v-$0.5$));\\
\\
//  Para pintar 'g', debe devolver un tipo 'vec$3$' y los dos primeros argumentos \\
// deben ser reales (u y v). El resto serán parámetros de tiempo (para homotopías)\\
plot g;\\
\hline
\end{tabular}
\end{center}

\newpage

\section{Especificación BNF}

Se describirá en esta sección de forma rigurosa el lenguaje que recibirá como entrada el programa "procesador":\\
\begin{tabular}{| p{3.5cm} c p{10cm} |}
\hline

<Programa>                &::=& <lista\_sentecias> <sentencia\_plot>; \\
&&\\

<lista\_sentecias>         &::=&  <lista\_sentecias> <sentencia>; \\
                           &|&  <sentencia>; \\
&&\\

<sentencia>               &::=&  <sentencia\_declar\_valor> \\
                           &|&  <sentencia\_declar\_fun> \\
&&\\

<sentencia\_declar\_valor>  &::=&  <identificador> : <identificador> = <expresion> \\
&&\\

<sentencia\_declar\_fun>    &::=&  <identificador>(<lista\_param>) : <identificador> = <expresion> \\
&&\\

<sentencia\_plot>          &::=&  plot <lista\_ident> \\
&&\\

<expresion>               &::=&  (<expresion>) \\
                           &|&  if <expresion> then <expresion> else <expresion> \\
                           &|&  <expresion> <op\_binario> <expresion> \\
                           &|&  <op\_unario> <expresion> \\
                           &|&  <identificador>(<lista\_expresiones>) \\
                           &|&  <expresion>[<expresion>] \\
                           &|&  <identificador> \\
                           &|&  <constante> \\
&&\\
                         
<op\_binario>              &::=& + |  - |  * |  / |  and |  or |  > |  < |  >= |  <= |  == |  != |  \^{} \\
&&\\

<op\_unario>               &::=&  - |  ! \\
&&\\

<lista\_expresiones>       &::=&  <lista\_expresiones>,<expresion> \\
                           &|&  <expresion> \\
&&\\

<lista\_param>             &::=&  <lista\_param>, <lista\_ident> : <identificador>  \\
                           &|&  <lista\_ident> : <identificador> \\
&&\\

<lista\_ident>             &::=&  <lista\_ident>, <identificador> \\
                           &|&  <identificador> \\
&&\\

<identificador>           &::=&  <letra><cadena> \\
                           &|&  <letra> \\
&&\\
                         
<cadena>                  &::=&  <cadena><letra> \\
                           &|&  <cadena><numero> \\
                           &|&  <letra> \\
                           &|&  <numero> \\
&&\\
\hline
\end{tabular}
\newpage
\begin{tabular}{| p{3.5cm} c p{10cm} |}
\hline
<constante>               &::=&  <numero\_entero> \\
                           &|&  <numero\_real> \\
                           &|&  <bool> \\
&&\\

<numero\_real>             &::=&  <numero\_entero> \\
                           &|&  <numero\_entero>.<numero\_entero> \\
&&\\

<numero\_entero>           &::=&  <numero\_entero><numero> \\
                           &|&  <numero> \\
&&\\

<bool>                    &::=&  true | false \\
&&\\

<letra>                   &::=&  a | ... | z | A | ... | Z \\
&&\\

<numero>                  &::=&  0 | 1 | 2 | 3 | 4 | 5 | 6 | 7 | 8 | 9 \\
&&\\
\hline
\end{tabular}


\endinput
%------------------------------------------------------------------------------------
% FIN DEL CAPÍTULO. 
%------------------------------------------------------------------------------------
