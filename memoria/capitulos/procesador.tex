% !TeX root = ../libro.tex
% !TeX encoding = utf8

\chapter{Procesador}

En este capítulo se describe el estudio realizado para teselar de manera eficiente un triángulo. Esta técnica se implementará en shaders, para poder ejecutarlo directamente en la GPU, que ofrece mayor rendimiento que la CPU para dicho problema. En un principio se iba a realizar en un Geometry Shader pero más tarde se observó que era más acertado el uso de un Tessellation Shader. \\
\\ Cabe destacar antes que la idea inicial era desarrollar un algoritmo de división recursiva de los triángulos, el cuál se detiene en el nivel en el que se cree que representa correctamente a la porción de superficie. Sin embargo, el lenguaje Glsl no permite realizar llamadas recursivas, por lo que era necesario buscar alternativas.

\section{Especificación BNF}


\begin{tabular}{| p{3.5cm} c p{10cm} |}
\hline

<Programa>                &::=& <lista\_sentecias> <sentencia\_plot>; \\
&&\\

<lista\_sentecias>         &::=&  <lista\_sentecias> <sentencia>; \\
                           &|&  <sentencia>; \\
&&\\

<sentencia>               &::=&  <sentencia\_declar\_valor> \\
                           &|&  <sentencia\_declar\_fun> \\
&&\\

<sentencia\_declar\_valor>  &::=&  <identificador> : <identificador> = <expresion> \\
&&\\

<sentencia\_declar\_fun>    &::=&  <identificador>(<lista\_param>) : <identificador> = <expresion> \\
&&\\

<sentencia\_plot>          &::=&  plot <lista\_ident> \\
&&\\

<expresion>               &::=&  (<expresion>) \\
                           &|&  if <expresion> then <expresion> else <expresion> \\
                           &|&  <expresion> <op\_binario> <expresion> \\
                           &|&  <op\_unario> <expresion> \\
                           &|&  <identificador>(<lista\_expresiones>) \\
                           &|&  <expresion>[<expresion>] \\
                           &|&  <identificador> \\
                           &|&  <constante> \\
&&\\
                         
<op\_binario>              &::=& + |  - |  * |  / |  and |  or |  > |  < |  >= |  <= |  == |  != |  \^{} \\
&&\\

<op\_unario>               &::=&  - |  ! \\
&&\\

<lista\_expresiones>       &::=&  <lista\_expresiones>,<expresion> \\
                           &|&  <expresion> \\
&&\\

<lista\_param>             &::=&  <lista\_param>, <lista\_ident> : <identificador>  \\
                           &|&  <lista\_ident> : <identificador> \\
&&\\

<lista\_ident>             &::=&  <lista\_ident>, <identificador> \\
                           &|&  <identificador> \\
&&\\

<identificador>           &::=&  <letra><cadena> \\
                           &|&  <letra> \\
&&\\
                         
<cadena>                  &::=&  <cadena><letra> \\
                           &|&  <cadena><numero> \\
                           &|&  <letra> \\
                           &|&  <numero> \\
&&\\
\hline
\end{tabular}
\newpage
\begin{tabular}{| p{3.5cm} c p{10cm} |}
\hline
<constante>               &::=&  <numero\_entero> \\
                           &|&  <numero\_real> \\
                           &|&  <bool> \\
&&\\

<numero\_real>             &::=&  <numero\_entero> \\
                           &|&  <numero\_entero>.<numero\_entero> \\
&&\\

<numero\_entero>           &::=&  <numero\_entero><numero> \\
                           &|&  <numero> \\
&&\\

<bool>                    &::=&  true | false \\
&&\\

<letra>                   &::=&  a | ... | z | A | ... | Z \\
&&\\

<numero>                  &::=&  0 | 1 | 2 | 3 | 4 | 5 | 6 | 7 | 8 | 9 \\
&&\\
\hline
\end{tabular}


\endinput
%------------------------------------------------------------------------------------
% FIN DEL CAPÍTULO. 
%------------------------------------------------------------------------------------
