% !TeX root = ../../libro.tex
% !TeX encoding = utf8

A continuación se enuncian los ``hechos'' utilizados para las demostraciones de los teoremas. Su demostración detallada en varios casos se escapa de los límites del proyecto, en cuyos casos se aportarán demostraciones que suponen ciertos aquellos elementos propios de la teoría de Morse.

\begin{hecho}
	Todo $W \subset \R^2$ abierto tiene una triangulación clásica tal que el tamaño de los triángulos se aproxima a $0$ en la frontera de $W$.
\end{hecho}

\begin{hecho}
	Toda variedad diferenciable $S$ tiene una triangulación diferenciable.
\end{hecho}

\begin{hecho}
	Para toda estructura diferenciable $E$ del toro punteado $T'_E$ existe un subconjunto compacto suyo cuyo complemento es difeomorfo a $S^1 \times \R$ con la estructura diferenciable usual.
\end{hecho}

\begin{hecho}
	Toda estructura diferenciable $E$ del toro $(S^1 \times S^1)_E$ es difeomorfa a la estructura usual del toro $S^1 \times S^1$.
\end{hecho}

\begin{hecho}
	Sea $E$ una estructura diferenciable en $D^1 \times \R$ tal que es la usual en un entorno del borde. Entonces existe un difeomorfismo $g:(D^1 \times \R)_E \rightarrow (D^1 \times \R)_U$, con $U$ la estructura usual, que además es la identidad entorno a $\partial D^1 \times \R$.
\end{hecho}

\begin{hecho}
	Sea $E$ una estructura diferenciable en $D^2$ tal que es la usual en un entorno del borde. Entonces existe un difeomorfismo $g:D^2_E \rightarrow D^2_U$, con $U$ la estructura usual, que además es la identidad entorno a $\partial D^2$.
\end{hecho}



\endinput
%------------------------------------------------------------------------------------
% FIN DEL CAPÍTULO. 
%------------------------------------------------------------------------------------
