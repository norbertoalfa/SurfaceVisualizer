% !TeX root = ../../libro.tex
% !TeX encoding = utf8

A continuación se enuncian los ``hechos'' utilizados para las demostraciones de los teoremas. Su demostración detallada en varios casos se escapa de los límites del proyecto, en cuyos casos se aportarán demostraciones que suponen ciertos aquellos elementos propios de la teoría de Morse.

\begin{hecho}
	Todo $W \subset \R^2$ abierto tiene una triangulación clásica tal que el diámetro euclídeo de los triángulos se aproxima a $0$ en la frontera topológica de $W$.
\end{hecho}

\begin{proof}
	Tomamos la cuadrícula generada de forma natural por $\Z^2$ sobre $\R^2$. Vamos a definir de forma incremental el conjunto de cuadrados que cubren todo $W$.\\
	\\ Tomamos primero todos los cuadrados (cerrados) que estén contenidos estrictamente en $W$. Vamos a llamar $U$ a la parte cubierta por el conjunto de cuadrados actual.\\
	\\ Dado un $p \in W$ y $p \not \in U$ entonces existe un cuadrado que lo contiene pero que no está contenido estrictamente en $W$. Por ser $W$ abierto sabemos que para $p$ existe una bola abierta $B \subset W$ que lo contiene. De forma equivalente podemos subdividir el cuadrado inicial en $4$ cuadrados iguales, $8$ ... y así sucesivamente hasta encontrar una subdivisión en la que algún cuadrado contenga a $p$ y sea lo suficientemente pequeño como para que esté contenido en la bola $B \subset W$. Añadimos al conjunto todos los cuadrados anteriores que estén contenidos en $W$. Realizamos esta operación de manera indefinida.\\
	\\ Una vez definido el conjunto de cuadrados, podemos definir la triangulación como los triángulos resultantes de dividir por la diagonal dichos cuadrados. De esta forma tenemos una triangulación $T$ tal que la unión de sus triángulos, $U$, está contenida en $W$ pero todo punto $p \in W$ está en algún triángulo, por lo que $U = W$. Además, cuando tomamos $p$ tendiendo a $\partial W$, la bola $B \subset W$ que lo contiene tiene radio $\epsilon$ tendiendo a $0$, es decir, el cuadrado necesario para cubrirlo correctamente tiende a $0$, y como consecuencia los dos triángulos de los que se compone también.
\end{proof}

\begin{hecho}
	Toda variedad diferenciable $S$ tiene una triangulación diferenciable.
\end{hecho}

\begin{proof}
	Vamos a contruir una malla de polígonos diferenciables, que nos dará paso de forma trivial a una malla de triángulos diferenciables.\\
	\\ Podemos tomar $f:S \rightarrow \R$ función de Morse apropiada, es decir, los inversos de compactos son compactos y todos sus puntos críticos están a distintos niveles (es posible porque están aislados). Cortamos $S$ por los niveles de puntos no críticos, para separar los puntos críticos entre sí, obteniendo así una descomposición de $S$ en piezas difeomorfas a:
	\begin{itemize}
		\item Discos, tiene un punto crítico de orden $0$ o $2$.
		\item Anillos, no tiene puntos críticos.
		\item ``Pantalones'', o de forma equivalente, medio toro al que se le ha quitado un disco en el interior. Tiene un punto crítico de orden $1$.
		\item ``Pantalones cruzados'', o también se pueden ver como medio toro suma conexa con un espacio proyectivo $\mathbb{RP}^2$. Por ello, se puede dividir en unos ``pantalones'' normales y una cinta de Möbius. Tiene un punto crítico de orden $1$.
	\end{itemize} 
	
	Por la teoría de Morse tenemos una división de $S$ en conjuntos difeomorfos a alguno de los anteriores, todos ellos pegados por circunferencias (los bordes de los conjuntos descritos). Podemos obtener una malla añadiendo un vértice a cada circunferencia y seguidamente si es:
	\begin{itemize}
		\item Un disco, se toma el centro y se divide el disco en $3$ partes, teniendo $3$ sectores difeomorfos a un triángulo.
		\item Un anillo, se unen los $2$ vértices (uno de cada circunferencia del borde) mediante un arco, obteniendo así un cuadrilátero.
		\item Unos ``pantalones'', se unen los $3$ vértices mediante $2$ arcos (un vértice común a los $2$ arcos), dando lugar a un heptágono.
		\item Una cinta de Möbius, se une el vértice con él mismo mediante un arco, el cual recorre la mitad de la cara de la cinta, obteniendo así un triángulo.
	\end{itemize} 
	
	Finalmente tenemos la malla de polígonos diferenciables, la cual podemos convertir en una triangulación diferenciable dividiendo de forma adecuada los polígonos.
\end{proof}

\begin{hecho}
	Para toda estructura diferenciable $E$ del toro punteado $T'_E$ existe un subconjunto compacto suyo cuyo complemento es difeomorfo a $S^1 \times \R$ con la estructura diferenciable usual.
\end{hecho}

\begin{hecho}
	Toda estructura diferenciable $E$ del toro $(S^1 \times S^1)_E$ es difeomorfa a la estructura usual del toro $S^1 \times S^1$.
\end{hecho}

\begin{hecho}
	Sea $E$ una estructura diferenciable en $D^1 \times \R$ tal que es la usual en un entorno del borde. Entonces existe un difeomorfismo $g:(D^1 \times \R)_E \rightarrow (D^1 \times \R)_U$, con $U$ la estructura usual, que además es la identidad entorno a $\partial D^1 \times \R$.
\end{hecho}

\begin{hecho}
	Sea $E$ una estructura diferenciable en $D^2$ tal que es la usual en un entorno del borde. Entonces existe un difeomorfismo $g:D^2_E \rightarrow D^2_U$, con $U$ la estructura usual, que además es la identidad entorno a $\partial D^2$.
\end{hecho}



\endinput
%------------------------------------------------------------------------------------
% FIN DEL CAPÍTULO. 
%------------------------------------------------------------------------------------
