% !TeX root = ../libro.tex
% !TeX encoding = utf8

\chapter{Análisis y diseño}

En este capítulo se especificará toda aquella información referente a la estructura del programa y los requisitos del mismo, aunque está mayormente enfocado a la definición de los algoritmos desarrollados.

\section{Especificación de requisitos}

	\subsection{Requisitos funcionales}
	\begin{enumerate}
		\item Se podrán visualizar varias parametrizaciones a la vez, donde cada una representará una carta de una superficie específica, con el objetivo de representar homotopías e isotopías entre superficies.
		\begin{enumerate}
			\item El sistema debe permitir visualizar cualquier parametrización que se le indique, siempre que cumpla con la estructura del lenguaje definido.
			\item Cada parametrización podrá admitir parámetros de ``tiempo'', para así modificar la porción de superficie que representa y así poder visualizar homotopías e isotopías.
		\end{enumerate}
		
		\item El programa contará con una interfaz clara, sencilla y completa.
		\begin{enumerate}
			\item El usuario tendrá la posibilidad de indicar manualmente los parámetros adicionales de las cartas $(t_i)$. Además se incluirá la opción de que cada $t_i$ se mueva de forma automática, para así generar animaciones fluidas.
			
			\item El usuario podrá indicar ciertos parámetros del cálculo de la malla de la superficie, como:
			\begin{enumerate}
				\item El tamaño de la malla inicial con la que se visualizará cada carta de la superficie.
				\item La precisión con la que se quiere representar la superficie actual.
			\end{enumerate}
				
			\item El usuario podrá indicar si quiere visualizar ciertos atributos de la superficie, como:
			\begin{enumerate}
				\item La curvatura de Gauss, asignando un color para la curvatura negativa y otro para la positiva, dependiendo de un parámetro de escala para resaltar las zonas.
				\item El área diferencial de la parametrización, junto con un umbral y un factor de escala que se podrán modificar.
				\item Se podrán visualizar los vectores tangente, normal y binormal de cada vértice generado.
			\end{enumerate}
			
			\item El usuario tendrá la posibilidad de modificar los valores referentes a la iluminación.
			\begin{enumerate}
				\item Los coeficientes del modelo de iluminación Phong.
				\item El color del fondo de la escena y el color base del objeto visualizado.
			\end{enumerate}
		\end{enumerate}
			
		\item El programa no renderizará nuevos frames si no se requieren nuevos cálculos, es decir, si no se detectan cambios en la entrada y la escena está estática.
	\end{enumerate}

	\subsection{Requisitos no funcionales}
	\begin{enumerate}
		\item El programa debe renderizar las superficies con un tiempo de respuesta bajo, pensando en dispositivos con una GPU común, como por ejemplo una gráfica integrada.
		\begin{enumerate}
			\item El programa adaptará su rendimiento según el estado del propio programa.
		\end{enumerate}
	\end{enumerate}

\section{Metodología de desarrollo}

\section{Diagramas}

\section{Principales desarrollos algorítmicos}

\endinput
%------------------------------------------------------------------------------------
% FIN DEL CAPÍTULO. 
%------------------------------------------------------------------------------------
