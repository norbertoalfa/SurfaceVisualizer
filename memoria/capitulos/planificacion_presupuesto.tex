% !TeX root = ../libro.tex
% !TeX encoding = utf8

\chapter{Planificación y presupuesto}

\begin{center}
\begin{tabular}{ | p{0.5cm} | p{2.5cm} | p{5cm} | p{5cm} | }
\hline
\multicolumn{4}{ | c | } {Planificación temporal en fases}\\
\hline
\multicolumn{1}{ | c | } {Nº} & \multicolumn{1}{ | c | } {Nombre} & \multicolumn{2}{ | c | } {Tareas}\\
\hline
$1$ & Inicial & \multicolumn{2}{ | p{10cm} | } {- Estudio inicial del problema}\\
\hline
$2$ & Implementación básica & Procesador: & Visualizador:\\
 & & - Definición del lenguaje (BNF) & - Toma de contacto con OpenGL, el lenguaje GLSL e ImGui.\\
 & & - Implementar primera versión del procesador a partir del código de la asignatura Procesadores de Lenguajes, que sólo detecta errores. & - Construir un programa de visualización a partir de código base, haciendo uso de los shaders básicos. La normal será la del triángulo.\\
 & & - Complementar el procesador para traducir a código GLSL (generación de árbol sintáctico), para usarlo en el vertex shader. & \\
\hline
$3$ & Primera conexión & \multicolumn{2}{ | p{10cm} | } {- Conectar apropiadamente los programas, para visualizar la superficie definida con la parametrización.} \\
 & & \multicolumn{2}{ | p{10cm} | } {- Estudiar el correcto funcionamiento de ambas aplicaciones hasta ahora. Corregir en caso de ser necesario.} \\
 & & \multicolumn{2}{ | p{10cm} | } {} \\
\hline
\end{tabular}
\end{center}

- Implementación media.
9.- Implementar los algoritmos que calculan los árboles de las derivadas parciales de las expresiones. Añadir seguidamente la generación de funciones típicas, como la normal, el área diferencial y la curvatura de Gauss.
10.- Utilizar la función normal en los puntos, en vez de las normales de los triángulos. Además, mostrar la normal como un segmento de color distinto al de la superficie.
11.- Mostrar mediante colores el valor del área diferencial y de la curvatura de Gauss en cada punto.
12.- Estudiar el correcto funcionamiento de ambas aplicaciones hasta ahora. Corregir en caso de ser necesario.

- Implementación avanzada.
13.- Implementar la teselación uniforme mediante el uso del geometry shader.
14.- Completar con una gran variedad de ejemplos.
15.- Estudiar el correcto funcionamiento del teselado.

- Estudio de la mejor medida.
16.- Elegir ciertas estadísticas del renderizado para comprobar si los resultados son buenos.
17.- Estudiar varias medidas para controlar el nivel de teselado.

- Estética
18.- Completar la interfaz del usuario.

- Revisión
19.- Estudiar el correcto funcionamiento de ambas aplicaciones hasta ahora. Corregir en caso de ser necesario.

\endinput
%------------------------------------------------------------------------------------
% FIN DEL CAPÍTULO. 
%------------------------------------------------------------------------------------
