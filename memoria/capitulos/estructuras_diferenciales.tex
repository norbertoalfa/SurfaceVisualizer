% !TeX root = ../libro.tex
% !TeX encoding = utf8

\chapter{Estructuras diferenciales}

\section{Resultados previos}

	\begin{teorema} (de ``alisamiento de asas'')
		Sea S una variedad diferenciable, entonces:
		\begin{enumerate}
			\item Un embebimiento $\mathbb{R}^2 \rightarrow S$ puede isotoparse a un embebimiento diferenciable entorno al origen, quedando fijo fuera de un entorno mayor al anterior.
			\item Un embebimiento $D^1\times\mathbb{R} \rightarrow S$ que es diferenciable entorno a $\partial D^1\times\mathbb{R}$ puede isotoparse a un embebimiento diferenciable entorno a $D^1\times 0$, quedando fijo fuera de un entorno mayor al anterior y cercano a $\partial D^1\times\mathbb{R}$.
			\item Un embebimiento $D^2 \rightarrow S$ que es diferenciable entorno a $\partial D^2$ puede isotoparse a un embebimiento diferenciable en todo $D^2$, quedando fijo en un entorno pequeño de $\partial D^2$.
		\end{enumerate}
	\end{teorema}

	\begin{proof}
		Voy a proceder a la demostración de cada uno de los apartados: \\
		\begin{enumerate}
			\item La idea de la demostración es inducir la estructura diferenciable de $S$ ($E_S$) al Toro ($T_S$) , obteniendo así un difeomorfismo de $T_S$ a la estructura diferenciable estándar del Toro (por el \textbf{Hecho 4}), que nos ayudará a construir la isotopía deseada.\\
				\\ Vamos a utilizar el ``truco del toro'', vemos el toro ($T$) como una variedad diferenciable y a su vez como el espacio de órbitas $\mathbb{R}^2/\mathbb{Z}^2$, tomando el $0$ como imagen del $0\in \mathbb{R}^2$. Eliminamos un punto del toro distinto del $0$, y a esta nueva variedad la llamamos $T'$. \\
				\\ Hacemos una inmersión $q: T' \rightarrow \mathbb{R}^2$ diferenciable manteniendo el $0$ y viendo $T'$ como el interior del disco unidad de $\mathbb{R}^2$ junto con dos 1-asas (las asas se embeben por separado ya que como se observa, se solapan en $\mathbb{R}^2$ y no es posible hacerlo junto).\\
				\\ Sea $h:\mathbb{R}^2 \rightarrow S$ embebimiento, por el cual $S$ induce una estructura diferenciable en $\mathbb{R}^2$, que denotaremos $E_1$. Por el mismo razonamiento (la inmersión definida para $T'$), $\mathbb{R}^2$ con la estructura $E_1$ induce una estructura diferenciable en $T'$, que llamaremos $E_2$.\\
				\\ Sabemos por el \textbf{Hecho 3} que existe un conjunto compacto en $T'_{E_2}$ cuyo complemento es difeomorfo a $S^1\times \mathbb{R}$, equivalentemente es difeomorfo a $D^2 - {(0,0)}$. Si lo vemos en el plano complejo, el $0$ se puede añadir de forma natural puesto que la estructura diferenciable usada hasta el momento es la usual en el cilindro (que induce la usual en $D^2$, en el plano complejo y en la esfera de Riemann). Esto nos permite extender la estructura diferenciable de $T'$ a $T$, llamada $E_2$.\\
				\\ Por el \textbf{Hecho 4} sabemos que toda estructura diferenciable del toro ($S^1\times S^1$) es difeomorfa a la estándar. Por tanto, existe un difeomorfismo $g: T_{E_2} \rightarrow T$. Para poder utilizar el Teorema de Lavantamiento de aplicaciones necesitamos normalizar dicha función $g$:
				\begin{itemize}
					\item Aplicando rotaciones en el toro (lo vemos como $S^1\times S^1$) podemos hacer que $g$ lleve el $0$ en el $0$.
					\item Necesitamos que $g_*$ sea la identidad para que al levantarla el $(0,0)$ de $\mathbb{R}^2$ vaya al $(0,0)$. Para ello, sabemos que existe un automorfismo diferenciable $M$ tal que $M_*$ pertenece a $GL_2(\mathbb{Z})$ y al componerlo con la actual $g$ nos lleva los generadores usuales del grupo fundamental del toro en ellos mismos (ya que $M_*$ lleva cualquier sistema de generadores en cualquier otro), es decir, la nueva $g_*$ es la identidad y $g$ sigue llevando el $0$ en el $0$.
				\end{itemize}
				 
				Podemos tomar el difeomorfismo $\widehat{g}:\mathbb{R}^2_{E_1} \rightarrow \mathbb{R}^2$ como el levantamiento de la normalización de $g$, que por la teoría de recubridores es periódico.\\
				\\ Identificamos $\mathbb{R}^2$ con el interior del disco unidad de $\mathbb{R}^2$ mediante una reparametrización radial que es la identidad entorno al $0$. Entonces aplicando esas identificaciones en el dominio y la imagen de $\widehat{g}$ obtenemos $G:D^2 \rightarrow D^2$ automorfismo diferenciable, que sigue siendo $\widehat{g}$ entorno al $0$ y tiende a ser la identidad en el borde (por la periodicidad $\|\widehat{g}(x) - x\|$ está acotado para todo $x$, y por consiguiente al tender $x$ a infinito las variaciones tienden a $0$ con la reparametrización, es decir, $G(x)$ tiende a $x$). Se puede extender a $G:\mathbb{R}^2 \rightarrow \mathbb{R}^2$, siendo la identidad fuera del interior del disco.\\
				\\ Por el truco de Alexander, $G$ es isotópica a la identidad. Se puede obtener la isotopía $G_t$ variando el radio del disco de origen y destino ($G_t(x) = tG(\frac{x}{t})$ para $x \in D((0,0), t)$ y es la identidad fuera), por lo que $G_0$ es la identidad y $G_1=G$.\\
				\\ Definimos la isotopía para $h$ como $h_t=h\circ G_t^{-1}$, teniendo que $h_0=h$ por ser $G_0$ la identidad. Además, $h_t$ se queda fija fuera del disco unidad ya que $G_t$ es la identidad en dicho conjunto. También tenemos que $h_t(0)=h(0)$ ya que para todo t, $G_t$ es la identidad entorno al $0$. Finalmente, $h_1$ es diferenciable entorno al $0$ porque $G_1^{-1}=G^{-1}$ es $\widehat{g}^{-1}$ entorno al $0$, que es un difeomorfismo de la estructura usual de $\mathbb{R}^2$ en la inducida por $S$ mediante $h$.
			\item La idea es, al igual que en el punto anterior, encontrar un difeomorfismo entre el dominio de $h$ con diferentes estructuras diferenciables y restringirlo para que esté fijo donde se solicite.\\
				\\ Tenemos $h: D^1\times \mathbb{R} \rightarrow S$ embebimiento diferenciable entorno al borde del dominio. Dicho embebimiento induce una estructura diferenciable en $D^1\times \mathbb{R}$ que al ser $h$ diferenciable entorno al borde, la estructura inducida es igual que la estructura estándar entorno al borde.\\
				\\ Por el \textbf{Hecho 5} tenemos que existe un difeomorfismo $g$ entre la estructura inducida y la estructura estándar de $D^1\times \mathbb{R}$ que es la identidad entorno al borde del conjunto. Tomamos el homeomorfismo $q: D^1\times \mathbb{R} \rightarrow (D^1\times D^1) - (0 \times \partial D^1)$ que es la identidad entorno $D^1\times 0$.\\
				\\ Definimos $G:  (D^1\times D^1) - (0 \times \partial D^1) \rightarrow (D^1\times D^1) - (0 \times \partial D^1)$ por $G = q \circ g \circ q^{-1}$, que como es la identidad entorno al borde del dominio, se puede extender a $G:\mathbb{R}^2 \rightarrow \mathbb{R}^2$. Su comportamiento entorno a $D^1 \times 0$ es igual que $g$ ($q$ es la identidad) y es la identidad fuera de $D^1 \times D^1$ y entorno a $\partial D \times \mathbb{R}$. Podemos tomar $G_t$ la isotopía, de forma que $t$ varía el tamaño del cuadrado $D^1 \times D^1$, de forma similiar a como se hacía en el apartado anterior para el círculo, teniendo así que $G$ es isotópica a la identidad.\\
				\\ Finalmente tomamos la isotopía $h_t = h \circ G_t^{-1}$, con $h_0 = h$ y $h_1$ tal que es $h$ donde $G$ es la identidad y es diferenciable en $D^1 \times 0$ (además de donde ya lo era $h$).
			\item Tenemos $h: D^2 \rightarrow S$ embebimiento que es diferenciable entorno al borde del dominio. Este embebimiento induce una estructura diferenciable de $S$ a $D^2$ que es igual que la estructura diferenciable estándar de $D^2$ entorno al borde, ya que $h$ es diferenciable en dicha zona.\\
				\\ Por el $\textbf{Hecho 6}$ existe un difeomorfismo $g$ entre $D^2$ con la estructura estándar y la inducida, que además es la identidad entorno al borde. El truco de Alexander nos aporta la isotopía $g_t$ donde $t$ va variando el tamaño del disco imagen de $g$ y extendiendo por la identidad, por lo que $g_0$ sería la identidad y $g_1 = g$. Tomando la isotopía $h_t = h \circ g_t^{-1}$ tendríamos lo solicitado, ya que $h_0 = h$ y $h_1$ es igual que $h$ en el borde y es diferenciable en todo el disco.
		\end{enumerate}
		
	\end{proof}
	\begin{corolario}
		El teorema anterior sigue siendo cierto para un abierto de $\mathbb{R}^2$ en vez de para todo $\mathbb{R}^2$.
	\end{corolario}


\section{Hechos utilizados para los teoremas}

\endinput
%------------------------------------------------------------------------------------
% FIN DEL CAPÍTULO. 
%------------------------------------------------------------------------------------
