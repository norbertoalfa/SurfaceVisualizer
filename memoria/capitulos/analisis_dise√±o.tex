% !TeX root = ../libro.tex
% !TeX encoding = utf8

\chapter{Análisis y diseño}

En este capítulo se especificará toda aquella información referente a la estructura del programa y los requisitos del mismo, aunque está mayormente enfocado a la definición de los algoritmos desarrollados.

\section{Especificación de requisitos}

	\subsection{Requisitos funcionales}
	\begin{enumerate}
		\item El sistema debe permitir visualizar cualquier parametrización que se le indique.
		\item Se podrán visualizar varias parametrizaciones a la vez, donde cada una podrá representar una carta de una superficie específica.
		\item El usuario podrá indicar ciertos parámetros de la visualización de la superficie, como:
			\begin{enumerate}
				\item El tamaño de la malla inicial con la que se visualizará cada carta de la superficie.
				\item El umbral de diferencia local de la longitud de la malla a generar con respecto a la superficie original.
			\end{enumerate}
		\item El programa no renderizará nuevos frames si no se requieren nuevos cálculos (no se detectan cambios en la entrada y la escena está estática).
	\end{enumerate}

	\subsection{Requisitos no funcionales}
	\begin{enumerate}
		\item El programa debe renderizar las superficies con un tiempo de respuesta bajo, pensando en dispositivos con GPU con características media, como por ejemplo gráficas integradas.
		\item Si el programa adaptará su rendimiento según el estado del propio programa.
	\end{enumerate}

\section{Metodología}

\section{Diagramas}

\section{Principales desarrollos algorítmicos}

\endinput
%------------------------------------------------------------------------------------
% FIN DEL CAPÍTULO. 
%------------------------------------------------------------------------------------
